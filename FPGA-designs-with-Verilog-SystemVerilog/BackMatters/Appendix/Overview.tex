\chapter{Script execution in Quartus and Modelsim}\label{QuartusModelsim}
\chapterquote{If you first fortify yourself with the true knowledge of the Universal Self, and then live in the midst of wealth and worldliness, surely they will in no way affect you.}{Ramakrishna Paramhansa}

\graphicspath{{Chapters/Overview/Figures/}}
\lstinputpath{Chapters/Overview/VerilogCodes/Overview/VerilogCodes}

To use the codes of the tutorial, Quartus and Modelsim softwares are discussed here. 

\section{Quartus}
In this section, `RTL view generation' and `loading the design on FPGA board' are discussed. 

\subsection{Generating the RTL view}
The execute the codes, open the `overview.qpf' using Quartus software. Then go to the files and right-click on the verilog file to which you want to execute and click on `Set as Top-Level Entity' as shown in Fig. \ref{fig:Quartus}. Then press `ctrl+L' to start the compilation. 

To generate the designs, go to \textbf{Tools$\rightarrow$Netlist Viewer$\rightarrow$RTL Viewer}; and it will display the design. 

\subsection{Loading design on FPGA board}
Quartus software generates two types of files after compilation i.e. `.sof' and `.pof' file. These files are used to load the designs on the FPGA board. Note that `.sof' file are erased once we turn off the FPGA device; whereas `.pof' files are permanently loaded (unless removed or overwrite manually). For loading the design on the FPGA board, we need to make following two changes which are board specific, 
\begin{enumerate}
	\item First, we need to select the board by clicking on \textbf{Assignments$\rightarrow$Device}, and then select the correct board from the list. 
	\item Next, connect the input/output ports of the design to FPGA board by clicking on \textbf{Assignments$\rightarrow$Pin Planner}. It will show all the input and output ports of the design and we need to fill `location' column for these ports. 
	
	\item To load the design on FPGA board, go to \textbf{Tools$\rightarrow$Programmer}. 
	\item Then select JTAG mode to load the `.sof' file; or `Active Serial Programming' mode for loading the `.pof' file. Then click on `add file' and  select the `.sof/.pof' file and click on `start'. In this way, the design will be loaded on FGPA board. 
\end{enumerate}


\begin{figure}[!h]
	\centering
	\includegraphics[scale=0.8]{Quartus}
	\caption{Quartus}
	\label{fig:Quartus}
\end{figure}


\section{Modelsim}
We can also verify the results using modelsim. Follow the below steps for generating the waveforms, 

\begin{enumerate}
	\item First, open the modelsim and click on `compile' button and select all (or desired) files; then press `Compile' and `Done' buttons. as shown in Fig. \ref{fig:Modelsim}.
	
	\begin{figure}[!h]
		\centering
		\includegraphics[width=\textwidth]{ModelsimSimulate}
		\caption{Modelsim: Compile and Simulate}
		\label{fig:Modelsim}
	\end{figure}
	
	\item Above step will show the compile files inside the `work library' on the library panel; then right click the desired file (e.g. comparator2Bit.vhd) and press `simulate',  as shown on the left hand side of the Fig. \ref{fig:Modelsim}. This will open a new window as shown in Fig. \ref{fig:ModelsimWave}. 
	
	\begin{figure}[!h]
		\centering
		\includegraphics[width=\textwidth]{ModelsimWave}
		\caption{Modelsim: Waveforms}
		\label{fig:ModelsimWave}
	\end{figure}
	
	\item  Right click the name of the entity (or desired signals for displaying) and click on `Add wave', as shown in Fig. \ref{fig:ModelsimWave}. This will show all the signals on the `wave default' panel. 
	
	\item Now go to transcript window, and write following command there as shown in the bottom part of the Fig. \ref{fig:ModelsimWave}. Note that these commands are applicable for 2-bit comparators only; for 1-bit comparator assign values of 1 bit i.e. `force a 1' etc. 
	\\
	\textbf{force a 00}
	\\
	\textbf{force b 01}
	\\
	\textbf{run}
	\\
	\\
	Above lines with assign the value 00 and 01 to inputs `a' and `b' respectively. `run' command will run the code and since `a' and `b' are not equal in this case, therefore `eq' will be set to zero and the waveforms will be displayed on `wave-default' window, as shown in Fig. \ref{fig:ModelsimWave}.  Next, run following commands, 
	\\
	\textbf{force a 01}
	\\
	\textbf{run}
	\\
	\\
	Now `a' and `b' are equal therefore `eq' will be set to 1 for this case. In this way we can verify the designs using Modelsim. 
	
\end{enumerate} 

%\begin{figure}
%	\centering
%	\includegraphics[width=\textwidth]{Modelsim}
%	\caption{Modelsim: Compile the designs}
%	\label{fig:Modelsim}
%\end{figure}



